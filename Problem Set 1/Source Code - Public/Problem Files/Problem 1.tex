\textbf{\hypertarget{P1}{[\,Suggested Time: 7 mins \textbar \, Total Marks: 7 \textbar \, Easy\,]}}\\
    A function f is defined by \(\displaystyle f:x\mapsto\sqrt{2}\left(\sin{x} + \cos{x} + 1\right)\) for \(0\le x\le 2\pi\)\\
    Points P and Q are the maximum and minimum points of f respectively,\\
    Without the use of calculus, find the coordinates P and Q \\
    \textbf{Leave your answers in exact values.} \qnmark{7}

%%%%%%%%%%%%%%%%%%
%%%%%Solution%%%%%
%%%%%%%%%%%%%%%%%%

%\begin{comment}

%%%%%Combining sin x and cos x with R-Formula%%%%%
\begin{gather*}
    \displaystyle Let \; y = \sqrt{2}\sin{x} + \sqrt{2}\cos{x} + \sqrt{2} \\
    \displaystyle Let \; R\sin{(x + \alpha)} = \sqrt{2}\sin{x} + \sqrt{2}\cos{x} \wrkonemark
\end{gather*}

\begin{align*}
    \displaystyle R\sin{\alpha} &= \sqrt{2} \\
    \displaystyle R\cos{\alpha} &= \sqrt{2}
\end{align*}

\begin{align*} %% Value of R
    \displaystyle R &= \sqrt{2+2} \\
                    &= 2 \wrkonemark
\end{align*}
\begin{align*} %% Value of \alpha
    \displaystyle \alpha &= \tan^{-1}{1} \\
                         &= \frac{\pi}{4} \wrkonemark
\end{align*}

%%%%%Rewriting the equation and finding P & Q%%%%%
\begin{gather*} %% Rewriting expression
    \displaystyle \implies\sqrt{2}\sin{x} + \sqrt{2}\cos{x} = 2\sin{\left(x + \frac{\pi}{4}\right)}\\
    \displaystyle \therefore y = 2\sin{\left(x + \frac{\pi}{4}\right)} + \sqrt{2}
\end{gather*}
\begin{multicols}{2}
    \begin{equation*}
        max(y) = 2 + \sqrt{2} \wrkonemark
    \end{equation*}
    \begin{align*} %% Coordinate of P
        \text{When} \; \sin{\left(x + \frac{\pi}{4}\right)} &= 1 \\
        \therefore x + \frac{\pi}{4} &= \frac{\pi}{2}\\
        x &= \frac{\pi}{4}
    \end{align*}

    \begin{equation*}
        \implies P = \left(\frac{\pi}{4} , 2 + \sqrt{2}\right) \ansonemark
    \end{equation*}

    \begin{equation*}
        min(y) = \sqrt{2} - 2 \wrkonemark
    \end{equation*}
    \begin{align*} %% Coordinate of Q
        \text{When} \; \sin{\left(x + \frac{\pi}{4}\right)} &= -1\\
        \therefore x + \frac{\pi}{4} &= \frac{3\pi}{2}\\
        x &= \frac{5\pi}{4}
    \end{align*}

    \begin{equation*}
        \implies Q = \left(\frac{5\pi}{4} , \sqrt{2} - 2\right) \ansonemark
    \end{equation*}
\end{multicols}

%\end{comment}