\textbf{\hypertarget{P8}{[\,Suggested Time: 25 mins \textbar \, Total Marks: 12 \textbar \, Challenging\,]}}\\
    A soft drink company wants to design a 500ml soda can using the least amount of \\
    material for their new Moon Shine soda. Assume that the soda can is a perfect \\
    cylinder. Using the information given below, find the cheapest material cost to \\
    produce 1000 cans. \textbf{Leave your answers in USD}  \qnmark{12} \\

%%%%%Information List%%%%%
\begin{center}
    \begin{tabularx}{0.65\textwidth} {
        | >{\centering\arraybackslash}X
        | >{\centering\arraybackslash}X| }
        \hline
        \multicolumn{2}{|c|}{\textbf{Information List}} \\
        \hline
        \textbf{Price of Aluminium} & \(2515 \; USD/Metric \; Tonne\) \\
        \hline
        \textbf{Density of Aluminium} & \(2.7g/cm^{3}\) \\
        \hline
        \textbf{1 Atmosphere (Pressure)} & 101325 Pa \\
        \hline
        \textbf{Room Temperature} & \(T_{c} = 28^{\circ}c\) \\
        \hline
        \textbf{Temperature (Kelvin)} & \(T_{k} = T_{c}+273\) K \\
        \hline
        \textbf{Gas Constant (R)} & \(R = 8.314\,m^{3}\,Pa/mol\,K\) \\
        \hline
    \end{tabularx}
    \\
    \vspace*{25pt}
    \begin{tabularx}{0.65\textwidth} {
        | >{\centering\arraybackslash}X| }
        \hline
        \textbf{Ideal Gas Law} \\
        \hline
        \textit{\textbf{PV = nRT} }\\*
        \(P - Pressure\) \\*
        \(V - Volume\) \\*
        \(n - Amount\;of\;Substance\) \\*
        \(R - Gas\;Constant\) \\*
        \(T - Temperature\) \\
        \hline
    \end{tabularx}
\end{center}

\vspace*{25pt}

    The Soda can is able to tolerate up to 5 atm of pressure, and has a uniform thickness \\
    of \(0.01\;cm\). The soda releases up to \(\displaystyle \frac{x}{10}\;cm^{3}\) of gas for every \(x\;ml\) of soda under room \\
    temperature and pressure.

\newpage

%%%%%%%%%%%%%%%%%%
%%%%%Solution%%%%%
%%%%%%%%%%%%%%%%%%

%\newpage \ \newpage \ \newpage

%\begin{comment}

%%%%%Formula for Surface area and Volume of Can%%%%%
Let \(S_{c}\;\&\;V_{c}\) be the total surface area and volume of the can respectively

\begin{align*}
    \displaystyle S_{c} &= 2 \pi rh+2\pi r^{2} \\
    \displaystyle V_{c} &= \pi r^{2}h
\end{align*}

%%%%%Finding the Volume of gas in the can at 5Pa (Max Pressure)%%%%%
    Room temperature and pressure \(\implies P = 1\;atm,\;T_{c} = 28^{\circ}c\) \\\\
    Thus, using the Ideal gas law, the number of moles of gas is
\begin{align*}
    \displaystyle n &= \frac{PV}{RT} \\
    \displaystyle   &= \frac{101325\left(\displaystyle \frac{50}{100^{3}}\right)}{8.314(273+28)} \\
    \displaystyle   &= 2.0245 \times 10^{-3} \wrkonemark \\
\end{align*}

    Since the max pressure that the can is able to handle is 5 atm, \\
    min volume taken up by gas is,
\begin{align*}
    \displaystyle V &= \frac{nRT}{P} \\
    \displaystyle   &= \frac{2.0245 \times 10^{-3}(8.314\times(273+28))}{5(101325)} \\
    \displaystyle   &= 1\times 10^{-5}\;m^{3} \\
    \displaystyle   &= 10\;cm^{3} \wrkonemark \\
\end{align*}
\textit{Thus, $V_{c} = 500 + 10 = 510 cm^{3}$}
\begin{align*}
    \displaystyle 510 &= \pi r^{2}h \\
    \displaystyle   h &= \frac{510}{\pi r^{2}} \wrkonemark
\end{align*}

%%%%%Finding the radius that gives the min. suf. area%%%%%
\begin{align*}
    \displaystyle  \therefore S_{c} &= 2\pi r\left(\frac{510}{\pi r^{2}}\right) + 2\pi r^{2} \\
    \displaystyle                   &= \frac{1020}{r} + 2\pi r^{2} \\
    \displaystyle \frac{dS_{c}}{dr} &= 4\pi r - \frac{1020}{r^{2}} \wrkonemark
\end{align*}

\newpage

When \(\displaystyle \frac{dS_{c}}{dr} = 0\)
\begin{align*}
    \displaystyle    4\pi r - \frac{1020}{r^{2}} &= 0 \wrkonemark \\
    \displaystyle \frac{4\pi r^{3} -1020}{r^{3}} &= 0 \\
    \displaystyle               4\pi r^{3} -1020 &= 0 \\
    \displaystyle \therefore               r^{3} &= \frac{1020}{4\pi} \\
    \displaystyle                              r &= \sqrt[3]{\frac{255}{\pi}} \approx 4.3298\;cm \wrkonemark
\end{align*}
\begin{align*} %%2nd Derivative Test
    \displaystyle                                                         \frac{d^{2}S_{c}}{dr^{2}} &= 4\pi + \frac{2040}{r^{3}} \\
    \displaystyle \left.\frac{d^{2}S_{c}}{dr^{2}}\right|_{\textstyle r = \sqrt[3]{\frac{300}{\pi}}} &= 4\pi + \frac{2040\pi}{255} \\
    \displaystyle                                                                                   &> 0 \wrkonemark
\end{align*}
\(\therefore \displaystyle r = \sqrt[3]{\frac{255}{\pi}}\) gives a min. amount of aluminium used

%%%%%Mass of aluminium required to make 1000 cans%%%%%
\begin{align*}
    \displaystyle V_{alu} &= \pi \left(0.01+\sqrt[3]{\frac{255}{\pi}}\right)^{2}\left(\frac{510}{\pi \left(\textstyle \sqrt[3]{\frac{255}{\pi}}\right)^{2}} + 0.02\right) - 510 \\
    \displaystyle         &\approx 3.5419\;cm^{2} \wrkonemark \\\\
\end{align*}
\begin{align*}
    \text{Mass of 1 Can} &= 3.5419(2.7) \\
                         &= 9.56313g \wrkonemark \\
    \text{Mass of 1k Cans}  &= 9.56313(1000) \\
                            &= 9563.13g \\
                            &\approx 9.5631kg \wrkonemark
\end{align*}

%%%%%Least material cost to produce 1000 cans%%%%%
\begin{align*}
    \displaystyle \text{Price of Aluminium in kg} &= \frac{2515}{1000} \\
                                                  &= 2.515 \; USD/kg \wrkonemark
\end{align*}
\begin{align*}
    \therefore \text{Material Cost to produce 1000 Cans} &= 9.5631(2.515) \\
                                                         &\approx 24.05 \; USD \ansonemark
\end{align*}
%\end{comment}

